\documentclass[11pt]{article}

\usepackage[utf8]{inputenc}
\usepackage{geometry}
\usepackage{color,graphicx,framed}
\usepackage{amsmath,amsfonts,amssymb}
\usepackage{listings}
\usepackage[section]{placeins}
\usepackage{pstricks,pst-tree,pst-node}
\usepackage{dsfont}
\usepackage{ulem}
\usepackage{tikz}
\usetikzlibrary{shapes,arrows,positioning,calc}

\title{Computergraphik SS14\\Übungsblatt 03}
\author{Björn Rathjen, Patrick Winterstein}
\date{zu 02.05.14}

\begin{document}
\maketitle
\newpage

\section*{10. Homogene Koordinaten (10 Punkte)}

\section*{12. Koordinatensysteme (12 Punkte, zu bearbeiten bis Donnerstag 7. Mai 2014)}
\begin{enumerate}
\item[(a)] Eine Kamera steht im Punkt
$\begin{pmatrix}
4 \\
5 \\
3 \\
\end{pmatrix}$
und blickt in Richtung auf den Punkt
$\begin{pmatrix}
7 \\
5 \\
4 \\
\end{pmatrix}$
Bestimmen Sie das entsprechende rechtwinklige Augenkoordinatensystem so, dass
die Kamera aufrecht steht.
\item[(b)] Bestimmen Sie die $4 \times 4$ - Transformationsmatrix zur Umrechnung von Weltkoordinaten in Augenkoordinaten. Sie sollen in der Lage sein, die Aufgabe auch mit abgeänderten Zahlen zu lösen.
\end{enumerate}
\section*{23. Projektive Ebene, homogene Koordinaten (0 Punkte)}
\begin{enumerate}
\item[(a)] Der Schnittpunkt von Geraden in homogenen Koordinaten kann mit dem Kreuz- produkt berechnet werden. Veranschaulichen und interpretieren Sie diese Formel am Raummodell der projektiven Ebene.
\item[(b)] Wie drückt sich die Tatsache, dass drei Punkte p 1 , p 2 , p 3 auf einer Geraden liegen, im Raummodell aus? Zeigen Sie, dass man diesen Sachverhalt an der folgenden Determinante ablesen kann:
$$\begin{pmatrix}
x 1 & x_2 & x_3 \\
y 1 & y_2 & y_3 \\
w 1 & w_2 & w_3 \\
\end{pmatrix}$$
\item[(c)] Bestimmen Sie eine projektive Transformation $x \rightarrow T x$ der Ebene, die die Gerade (1, 2, 3) auf die Ferngerade (0, 0, 1) abbildet und den Ursprung (0, 0, 1) festhält. Ist diese Transformation eindeutig?
\end{enumerate}
\section*{24. Rendering pipeline (18 zusätzliche Punkte), Erweiterung von Aufgabe 12.}
Die 8 Ecken eines Würfels haben in seinem lokalen Koordinatensystem die Koordinatem (±1, ±1, ±1). In Weltkoordinaten wird der Würfel mit den Faktor 0.2 skaliert, um 60 um die vertikale Achse (die z-Achse) gedreht, und zwar nach links (im Gegenuhrzeigersinn), wenn man von oben (aus der positiven z-Richtung) draufschaut. Sein Zentrum liegt bei (x, y, z) = (9, 6, 5). Eine Kamera auf dem Punkt (4, 5, 3) blickt mit einem horizontalen Öffnungswinkel von 30 ◦ in Richtung des Punktes (7, 5, 4). Die Oben-Richtung der Kamera ist dabei möglichst senkrecht. Die Ausgabe der Kamera erscheint auf einem 600×400 Bildschirm. Der Hauptpunkt ist in der Mitte des Bildschirmrechtecks. Wählen Sie die vordere und die hintere Begrenzung des sichtbaren Kegelstumpfs im Abstand von 3 und 20 Einheiten. Beschreiben Sie die notwendigen Rechnungen in den einzelnen Stufen der rendering pipeline und führen Sie sie aus, um die Bildschirmkoordinaten (Bildpunkte) der acht Würfelecken zu bestimmen. Zeichnen Sie Ihr Ergebnis als Skizze. Geben Sie auch die 4 × 4-Transformationsmatrix vom lokalen Koordinatensystem des Würfels auf die (dreidimensionalen) normalisierten Bildschirmkoordinaten (NDC) im Würfel [−1, 1] 3 an.


\end{document}