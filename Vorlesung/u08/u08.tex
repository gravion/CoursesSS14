\documentclass[11pt]{article}

\usepackage[utf8]{inputenc}
\usepackage{geometry}
\usepackage{color,graphicx,framed}
\usepackage{amsmath,amsfonts,amssymb}
\usepackage{listings}
\usepackage[section]{placeins}
\usepackage{pstricks,pst-tree,pst-node}
\usepackage{dsfont}
\usepackage{ulem}
\usepackage{nicefrac}

\title{Computergrafik\\Übungsblatt 08}
\author{Björn Rathjen}
\date{SS14}

\begin{document}
\maketitle
\newpage
\section{Aufgabe 42 : Interpolation}
\begin{itemize}
\item[(a)]
 \begin{eqnarray}
P' &=& T P \\
&=& T ( \frac{1}{3} A + \frac{1}{3} B + \frac{1}{3} C \\
&=& T \frac{1}{3} A + T\frac{1}{3} B + T\frac{1}{3} C \\
&=& \frac{1}{3} TA + \frac{1}{3} TB + \frac{1}{3} TC \\
&=& \frac{1}{3} \begin{pmatrix}
9 \\ 0 \\ 35 \\49 
\end{pmatrix} + \frac{1}{3} 
\begin{pmatrix}
15 \\ -3 \\ 3 \\ 12 
\end{pmatrix} + \frac{1}{3}
\begin{pmatrix}
18 \\ 36 \\ 47 \\ 5 \\
\end{pmatrix} \\
&=& \frac{1}{3} \begin{pmatrix} 9 + 15 + 18\\
0 - 3 + 36 \\
35 + 3 + 47 \\
49 + 12 + 5 \\
\end{pmatrix}\\
&=& \frac{1}{3} \begin{pmatrix}
42 \\ 33 \\ 85 \\ 66 
\end{pmatrix}\\
&=& \begin{pmatrix}
14 \\ 11 \\ \frac{85}{3} \\ 22 
\end{pmatrix}\\
&=& \begin{pmatrix}
\nicefrac{14}{22} \\ \nicefrac{11}{22} \\ \nicefrac{85}{66} \\ 1 
\end{pmatrix}\\
&\Rightarrow &  \text{Karthesische Koordinaten}
\begin{pmatrix}
\nicefrac{7}{11} \\ \nicefrac{1}{2} \\ \nicefrac{85}{66}
\end{pmatrix}
\end{eqnarray}
\newpage
\item[(b)]
Vorraussetzung $j_2 > j_1$
\begin{lstlisting}
func nocmalvektor(){
	list normalenvektor
	i = 0
	dj = j_2 - j_1
	dn = n - n'
	step =  dn / dj
	for i=0; i < dj; i++ {
		normalev = (0.6 , 0.8 ,0) + i * step
		Length = sqrt(normalv(1)^2 + normalv(2)^2 + normalv(3)^2)
		normalenvektoren.add( normalv/ Lenght )
	} 
	return normalenvektoren
}
\end{lstlisting}
es fe
\item[(c)] Es muss jeweils die Position von I interpoliert werden. Dies geschieht ebenfalls über die Eckpunkte der entstandenen Dreiecke. 
\end{itemize}
\end{document}